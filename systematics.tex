\section{Systematics}
\subsection{Nuclear Dependence of $R=\sigma_L/\sigma_T$}


Due to the relatively low energy of the 6~GeV JLab E03-103 measurements, effects due to acceleration and
deceleration of electrons in the Coulomb field of the heavier targets (Cu and Au) could not be ignored.
However, once applied, these so-called Coulomb Corrections resulted in ratios systematically larger
than those found by previous experiments.  This apparent discrepancy motivated the re-examination of
earlier measurements, and it was found that, while the bulk of the large $x$ measurements from SLAC
were taken at significantly higher beam energies, SLAC E140 (an experiment dedicated to studying
the nuclear dependence of $R=\sigma_L/\sigma_T$) had used beam energies similar to JLab E03-103, but for
those measurements had not applied Coulomb Corrections.  Once Coulomb Corrections were applied to the E140
results, the conclusion that there was no evidence for a nuclear dependence of $R$ was less strong, with
$\Delta R=R_A-R_D$ about 1.5 standard deviations from zero. Further, when a combined analysis
of available SLAC E139, E140, and JLab E03103 data was performed for data at $x=0.5$ and $Q^2\approx5$~GeV$^2$,
a similar deviation of $\Delta R$ was found~\cite{Solvignon:2009it}.

This result, combined with hints of a difference in $R$ for protons and deuterium motivated a 
new experiment to make further measurements of the nuclear dependence of $R=\sigma_L/\sigma_T$
with better precision than E140 and for a wider range of $x$ and $Q^2$ than previously
measured~\cite{12gev_nucr}.  This experiment will provide precise measurements of $R=\sigma_L/\sigma_T$
for the nucleon for $0.1<x<0.6$ and $1<Q^2<5$ as well as determination of $\Delta R= R_A-R_D$ for the same
kinematics using a copper target. Additional data will also be taken with carbon and gold targets for a subset
of the kinematics.




