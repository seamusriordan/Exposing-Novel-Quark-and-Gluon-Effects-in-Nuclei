
%THIS PART, add in the first paragraph of ``NPDFS IN THE HIGH-X REGION", between ``... available nuclei." and ``Extrapolations ..."

%Less about one third of the data used in the fits come from heavy nuclei which complicates the possibility of truly separating the nuclear modifications for each flavour. A possible path for doing so would be using charged current (CC) data from DIS, available for $Fe$ and $Pb$, as the cross-section depend on different combinations of the PDFs than the neutral current (NC) processes. It has been also suggested that nuclear effects might not be universal and therefore making a truly global fit of the nPDFs would not be possible. Up to now and within experimental uncertainties, NLO fits including NC and CC data have not shown visible tension. Unfortunately these fixed target experiments cover a very limited region of the kinematic space and are lacking in precision, not allowing yet for a conclusive answer. 

\subsection{Improvements for nPDFs}

Future experiments will play a key role on determining the nPDFs. One attractive feature of these is the possibility of incorporating lessons learned from past experiments into the data extraction, publication and analysis. A simple example of this are the results from HERA where initial data were published in the form of the structure functions $F_{2}$, while in later years, and after the conclusion of data taking, new studies using the cross-sections and combining H1 and ZEUS results have become available. The realization came in that the structure functions are not a fundamental observable when considering the gluon densities and potential saturation effects. This adds uncertainty due in the $F_{2}$ and no reanalysis under this light has been performed.  Assumptions are made about the form of the longitudinal structure functions $F_\mathrm{L}$ and become deeply ingrained in the results.  Similarly, the corrections based on explicit assumptions on the flavor dependence of the modification in nuclear data analyses which account for the non-isoscalarity of some targets, can lead to very different shapes for the EMC effect.  In this light it would be extremely beneficial for the community to publish future results which include  the measured $\sigma$, along side the structure functions, with and without corrections or assumptions. 

%Another aspect that would be beneficial for the extraction of the nPDFs is the improvement of our understanding of final state effects. Unlike proton PDFs, where the determination of the gluon is not contaminated by final states, the nPDFs have been including routinely particle production in $d+Au$ collisions into the fits, an observable that is very sensitive to both the initial and final state (hadronization) of the gluon. This would be in principle a perfectly adequate idea, except that it has some caveats. The fragmentation functions (FFs) in the vacuum are best determined for the pions, with the gluon fragmentation having a sizable theoretical uncertainty. Moreover data from the LHC at $7$ GeV can not be well described by a global fit unless a significant cut in $p_{T}$ is applied to the date, leaving out a relevant portion of the $d+Au$ covered region. Furthermore, when it comes to particle production in a nuclear medium it is yet unclear how much of the measured effects can be attributed to the nPDFs and how much of it, if any, is a final state effect due to the partons interacting with the medium before the hadronization. The only semi-inclusive DIS off nuclei, measured at HERMES, shows a significant effect that has yet to be understood. This constitutes a sizable source of uncertainty for the extraction of the nPDFs and therefore conclusions from incorporating particle production data form RHIC into the fits must be drawn carefully.   

