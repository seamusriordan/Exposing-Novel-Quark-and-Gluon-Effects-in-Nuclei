\section{Nuclear Dependence of $R=\sigma_L/\sigma_T$}
\label{sec:long}

Due to the relatively low energy of the 6~GeV JLab E03-103 measurements, effects due to acceleration and
deceleration of electrons in the Coulomb field of the heavier targets (Cu and Au) could not be ignored.
However, once applied, these so-called Coulomb corrections resulted in ratios systematically larger
than those found by previous experiments.  This apparent discrepancy motivated the re-examination of
earlier measurements, and it was found that, while the bulk of the large $x$ measurements from SLAC
were taken at significantly higher beam energies, SLAC E140 (an experiment dedicated to studying
the nuclear dependence of $R=\sigma_L/\sigma_T$, where $\sigma_L$ and $\sigma_T$ are the longitudinally and transversely polarized virtual photon component of the inelastic cross section) had used beam energies similar to JLab E03-103, but for
those measurements had not applied Coulomb corrections.  Once Coulomb corrections were applied to the E140
results, the possibility for concluding a nuclear dependence of $R$ was stronger but not unambiguous, with
$\Delta R=R_A-R_D$ about 1.5 standard deviations from zero. Further, when a combined analysis
of available SLAC E139, E140, and JLab E03-103 data was performed for data at $x=0.5$ and $Q^2\approx5$~GeV$^2$,
a similar deviation of $\Delta R$ was found~\cite{Solvignon:2009it}. Dedicated, high precision longitudinal/transverse separation experiments performed on hydrogen and deuterium at JLab also observed a likely nuclear dependence to $R=\sigma_L/\sigma_T$~\cite{Tvaskis:2016uxm,Tvaskis:2006tv}. Subsequent experiments on heavy nuclei in the resonance region at JLab, where effects have been predicted~\cite{Miller:2001yf} clearly demonstrated a nuclear dependence to $R${~\cite{Keppel_private} in this regime, although different from expected. 

The above observations and results, performed where $R$ is relatively large and therefore a nuclear dependence is measurable (as compared to previous high Q$^2$ measurements) motivated a 
new experiment to make further measurements of the nuclear dependence of $R=\sigma_L/\sigma_T$
with high precision and covering a wider range of $x$ and $Q^2$ than previously
measured~\cite{12gev_nucr}.  This experiment will provide precise measurements of $R=\sigma_L/\sigma_T$
for the nucleon for $0.1<x<0.6$ and $1<Q^2<5~\mathrm{GeV}^2$ as well as determination of $\Delta R= R_A-R_d$ for the same
kinematics using a copper target. Additional data will also be taken with C and Au targets for a subset
of the kinematics. Unanswered questions such as if the EMC effect is a longitudinal or transverse or combined effect will be answered by this measurement where, thus far, it has been incorrectly assumed that the longitudinal contribution is negligible. An analysis of the impact of the projected data points out that, in the presence of the observed small but non-zero difference between $R$ for nuclei and the nucleon, the nuclear enhancement in the ratio of the transverse structure functions $F_1^A/F_1^D$ becomes significantly reduced (or even disappears in some cases), indicating that anti-shadowing is dominated by the longitudinal contribution~\cite{Guzey:2012yk}. 

$R$, and correspondingly, the longitudinal nuclear inclusive structure function $F_L^A(x,Q^2)$, are quantities that directly probe the nuclear gluon distribution $g_A(x)$ in leading twist framework. The magnitude of nuclear enhancement to these structure functions is directly correlated with the size and shape of the nuclear gluons. While at the moment $g_A(x)$ is rather poorly constrained by QCD fits to available data, dedicated high-precision measurements of the nuclear dependence of $R$ at JLab and the EIC have the potential to constrain $g_A(x)$ in the anti-shadowing and EMC regions and beyond. Through the parton momentum sum rule, this knowledge will have some impact on $g_A(x)$ over the entire range of x, and so should also help to constrain $g_A(x)$ in the nuclear shadowing region, where the gluon distribution plays an essential role in the phenomenology of high-energy hard processes with nuclei. 




