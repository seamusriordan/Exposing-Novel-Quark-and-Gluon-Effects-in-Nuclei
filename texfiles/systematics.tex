\section{Systematics\label{sec:systematics}}


\subsection{Hadronization and Final State Effects}

The formation of hadrons and the propagation of quarks in a nuclear medium is important for interpreting the final states of reactions in semi-inclusive DIS and Drell-Yan processes as well as critical for interpreting gluon distributions.  An important open question are the relative sizes of the interactions between asymptotic quark propagation and interactions after hadronization.  As final states can reveal important information regarding the kinematics (such as Bjorken $x$) and flavor in a given interaction, improving the quality studies and models will simultaneously improve the extraction of modification data.  Studies have taken place at a variety of facilities, e.g. Fermilab~\cite{PhysRevC.75.035206}, JLab~\cite{PhysRevLett.99.242502, ELFASSI2012326}, and HERMES~\cite{Airapetian2011} as well as a future program with CLAS12 at JLab to study this in a broad set of channels~\cite{quarkformprop}.

Determination of the gluon distributions are also contaminated by final state interactions. nPDFs have been including routinely particle production in $d+\mathrm{Au}$ collisions into the fits, an observable that is very sensitive to both the initial state and hadronization of the gluon. The fragmentation functions for pion production have been best determined from $e^+e^-$ data, though the fragmentation of gluons have a sizable theoretical uncertainty. Data from the LHC at $7$~TeV can not be well described by a global fit unless a significant cut in $p_{T}$ is applied to the data, leaving out a relevant portion of the covered $d+\mathrm{Au}$ region.  With the present data, this constitutes a sizable source of uncertainty for the extraction of the nPDFs and conclusions from incorporating particle production data from hadron colliders into the fits must be drawn carefully.

\subsection{Free Nucleon Parton Distributions}

While the free proton parton distributions have been studied extensively and with great precision, there still remain important measurements to be done to constrain the two leading-flavor parton distributions, in particular in the ratio of $d/u$ limit as $x \rightarrow 1$.  As these represent the basis of comparison for any nuclear modification effect, it is critical to have high quality data available, especially as one considers doing flavor decompositions of nuclei.  There are several programs which intend to improve the fixed-target lepton scattering data, such as using the ratio of ${}^{3}$H and ${}^{3}$He cross section~\cite{mar}, tagged spectator with deuterium~\cite{bonus12}, and parity-violating deep inelastic scattering on the proton~\cite{solid_pvdis}.  In addition, recent analyses of $W$ and $Z$ production in $p\bar{p}$ collisions from the CDF and D\O\ collaborations~\cite{D0:2014kma,Abazov:2013dsa,Acosta:2005ud,Aaltonen:2009ta,Aaltonen:2010zza,Abazov:2007jy} have also provided new constraints at large $x$ and have been incorporated into a recent global PDF analysis~\cite{Accardi:2016qay}.

